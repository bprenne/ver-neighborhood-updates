\listfiles
\documentclass[12pt]{article}

\setlength{\parindent}{0mm}
\setlength{\parskip}{0.5em}

%%
%% page setup and comments:
%% - comment out the \def\comments{1} if you do not want comments
%% - leave the \def\comments{1} uncommented if you do want comments
%%

\def\comments{1}

%% leave the next few lines below alone

\ifx\comments\undefined
  % then do normal margins and ignore comments
  \usepackage[a4paper,margin=4cm,includefoot]{geometry}
  \newcommand{\XXXcomment}[1]{}
\else
  % then do big marginpar margins and insert comments
  \usepackage[a4paper,margin=.5in,right=2.5in,marginpar=2in,includefoot]{geometry}
  \newcommand{\XXXcomment}[1]{\marginpar{\color{blue}{\footnotesize #1}}}
\fi

%%
%% packages and package-specific settings
%%

\usepackage{latexsym,amssymb,amsmath,amscd,amsthm,alltt,stmaryrd,graphicx,mathrsfs}

\usepackage{times}

\usepackage{float}
\floatstyle{boxed}
\restylefloat{figure}
\restylefloat{table}

%%
%% tikz stuff, including custom environment
%%

\usepackage{tikz}
\usetikzlibrary{trees,arrows,positioning,patterns,automata,shapes,decorations,decorations.pathmorphing}

\newcommand{\mystretch}{\renewcommand{\arraystretch}{1.5}}
\newcommand{\normalstretch}{\renewcommand{\arraystretch}{1}}

\newlength{\mysep}
\setlength{\mysep}{2.7cm}
\newlength{\mysepCloser}
\setlength{\mysepCloser}{2cm}
\newlength{\mysepFarther}
\setlength{\mysepFarther}{3cm}
\newlength{\mysepFarthest}
\setlength{\mysepFarthest}{4cm}

\newlength{\myMinsize}
\setlength{\myMinsize}{3em}
\newlength{\myBigger}
\setlength{\myBigger}{4em}
\newlength{\myNodeDistance}
\setlength{\myNodeDistance}{3cm}

\newenvironment{mytikz}[1][0em]{
\begin{tikzpicture}[>=latex,auto,node distance=\mysep,
  baseline={([yshift=#1]current bounding box.east)}]

  \normalstretch{}

  \tikzstyle{w}=[draw,circle,thick,minimum size=\myMinsize]

  \tikzstyle{e}=[draw,minimum size=\myMinsize,node distance=\myNodeDistance]
  
  \tikzstyle{every edge}=[draw,thick,font=\footnotesize]
  
  \tikzstyle{every label}=[font=\footnotesize]
  
  \tikzstyle{ev}=[anchor=west,node distance=\myNodeDistance]

  \tikzstyle{bigger}=[minimum size=\myBigger]

  \tikzstyle{closer}=[node distance=\mysepCloser]  

  \tikzstyle{farther}=[node distance=\mysepFarther]

  \tikzstyle{farthest}=[node distance=\mysepFarthest]

  \tikzstyle{l}=[node distance=\myNodeDistance]
}{\mystretch{}\end{tikzpicture}}

%%
%% theorem environments
%%

\theoremstyle{definition}
\newtheorem{theorem}{Theorem}[section]
\newtheorem{proposition}[theorem]{Proposition}
\newtheorem{definition}[theorem]{Definition}
\newtheorem{lemma}[theorem]{Lemma}
\newtheorem{conjecture}[theorem]{Conjecture}
\newtheorem{example}[theorem]{Example}
\newtheorem{question}[theorem]{Question}
\newtheorem{exercise}[theorem]{Exercise}
\newtheorem{remark}[theorem]{Remark}

%%
%% custom commands
%%

\newcommand{\Nat}{\mathbb{N}}  % natural numbers
\newcommand{\Rat}{\mathbb{Q}}  % rational numbers
\newcommand{\Ree}{\mathbb{R}}  % real numbers
\newcommand{\Int}{\mathbb{Z}}  % integers

\newcommand{\pow}{{\cal P}}    % powerset

\newcommand{\M}{{\cal M}}      % caligraphic model M
\newcommand{\N}{{\cal N}}      % caligraphic model N

\newcommand{\Prop}{{\bf P}}    % propositional letters

\newcommand{\Lang}{{\cal L}}   % language

\newcommand{\conv}{\check{\ }} % dual modality

\newcommand{\pre}{\mathsf{pre}}
\newcommand{\Pref}{{\bf P}}

\newcommand{\A}{{\bf A}}
\newcommand{\B}{{\bf B}}

\newcommand{\Actm}{\mathfrak{A}}   % action models

\newcommand{\KB}{{\mathsf{KB}}}                        % base theory
\newcommand{\KBlt}{{\mathsf{KB}^{\mathsf{<.5}}}}       % theory for c<.5
\newcommand{\KBeq}{{\mathsf{KB}^{\mathsf{=.5}}}}       % theory for c=.5
\newcommand{\KBeqm}{{\mathsf{KB}^{\mathsf{=.5}}_{-}}}  % theory for c=.5 without (M)
\newcommand{\KBgt}{{\mathsf{KB}^{\mathsf{>.5}}}}       % theory for c>.5
\newcommand{\KBgeq}{{\mathsf{KB}^{\mathsf{\geq.5}}}}   % theory for c>=.5
\newcommand{\KBc}{{\mathsf{KB}^{\mathsf{>}c}}}         % theory for c

\newcommand{\KBmyc}[1]{{\mathsf{KB}^{\mathsf{>{#1}}}}} % theory for specific c = #1


\newcommand{\Clt}{{\mathcal{C}^{\mathsf{<.5}}}}    % neigh. models for c<.5
\newcommand{\Ceq}{{\mathcal{C}^{\mathsf{=.5}}}}    % for c=.5
\newcommand{\Cgt}{{\mathcal{C}^{\mathsf{>.5}}}}    % for c>.5

\newcommand{\Rel}[1]{\stackrel{#1}{\Longrightarrow}}         % #1 over \Longrightarrow

\newcommand{\sem}[1]{\llbracket{#1}\rrbracket}               % [[ #1 ]]

\newcommand{\modelsn}{\models_{\mathsf{n}}}                  % \models_n
\newcommand{\semn}[1]{\llbracket{#1}\rrbracket_{\mathsf{n}}} % [[ #1 ]]_n

\newcommand{\modelsp}{\models_{\mathsf{p}}}                  % \models_p
\newcommand{\semp}[1]{\llbracket{#1}\rrbracket_{\mathsf{p}}} % [[ #1 ]]_p

%%
%% author, title, &c
%%

\newcommand{\ourtitle}{Belief Revision By Neighborhood Update}

\newcommand{\jan}{Jan van Eijck}

\newcommand{\janAffiliation}{CWI \&\ ILLC, Amsterdam}

\newcommand{\bryan}{Bryan Renne}

\newcommand{\bryanAffiliation}{ILLC, University of Amsterdam}

\newcommand{\bryanFunding}{Funded by an Innovational Research
  Incentives Scheme Veni grant from the Netherlands Organisation for
  Scientific Research (NWO).}

\title{\ourtitle{}} 

\author{\jan{}\\{}{\small\janAffiliation{}} \and
  \bryan{}\footnote{\bryanFunding{}}\\{}{\small\bryanAffiliation{}}}

\usepackage[pdftex,
            bookmarks=true,
            bookmarksnumbered=true,
            pdfborder={0 0 0},
            plainpages=false,
            pdfpagelabels]{hyperref}

\hypersetup{ 
  pdfauthor={\jan{}, \bryan{}}, 
  pdftitle={\ourtitle{}}
}

%%
%% beginning of paper
%%

\begin{document}

\maketitle

\begin{abstract}
  \noindent 
  We study update of neighborhood models for belief with neighborhood
  models for communication and change. Our main motivation for this is
  that the update construction for such model is better behaved than
  the product update of Kripke models for belief with Kripke event
  models.
\end{abstract}

\section{Introduction}

\ldots 

\section{Epistemic Neighborhood Models} 

Let $A$ be a finite set of agents and $\Prop$ a set of basic propositions. 

\begin{definition} 
  An \emph{epistemic neighborhood model} is a structure
  \[
  \M=(W,R,V,N)
  \]
  satisfying the following.
  \begin{itemize}
  \item $(W,R,V)$ is a finite multi-agent $\mathsf{S5}$ Kripke model, 
    i.e., $W$ is an finite set of worlds, $R$ assigns to every agent 
    $a$ an equivalence relation $R_a$ on $W$, and $V$ assigns to  
    every $w \in W$ a subset of the set of basic propositions $\Prop$. 
    We let 
    \[
    [w]_a:=\{v\in W\mid wR_av\}
    \]
    denote the $a$-equivalence class of world $w$.  This is the
    set of worlds $a$ cannot distinguish from $w$.

  \item $N : A\times W \to \pow(\pow(W))$ is a \emph{neighborhood
      function} that assigns to each agent $a\in A$ and world $w\in W$
    a collection $N_a(w)$ of sets of worlds---each such set called a
    \emph{neighborhood} of $w$---subject to the following conditions.
    \begin{description}
    \item[(kbc)] $\forall X \in N_a(w) : X \subseteq [w]_a$.

    \item[(bf)] $\emptyset\notin N_a(w)$.
      
    \item[(n)] $[w]_a\in N_a(w)$.
      
    \item[(a)] $\forall v \in [w]_a : N_a(v) = N_a(w)$.

    \item[(kbm)] $\forall X \subseteq Y \subseteq [w]_a : 
      \text{ if } X \in N_a(w) \text{, then } Y \in N_a(w)$.
   \end{description}
  \end{itemize}
  A \emph{pointed epistemic neighborhood model} is a pair $(\M,w)$
  consisting of an epistemic neighborhood model $\M$ and a world $w$
  in the set of worlds of $\M$. 
\end{definition}

An epistemic neighborhood model is a variation of a neighborhood model that
includes an epistemic component $R_a$ for each agent $a$.
Intuitively, $[w]_a$ is the set of worlds agent $a$ knows to be
possible at $w$ and each $X\in N_a(w)$ represents a proposition that
the agent believes at $w$.  The condition that $R_a$ be an equivalence
relation ensures that knowledge is closed under logical consequence,
veridical (i.e., only true things can be known), positive
introspective (i.e., the agent knows what she knows), and negative
introspective (i.e., the agent knows what she does not know).

Property (kbc) ensures that the agent does not believe a proposition
$X\subseteq W$ that she knows to be false: if $X$ contains a world in
$w'\in(W-[w]_a)$ that the agent knows is not possible with respect to
the actual world $w$, then she knows that $X$ cannot be the case and
hence she does not believe $X$.  Property (bf) ensures that no logical
falsehood is believed, while Property (n) ensures that every logical
truth is believed.  Property (a) ensures that if $X$ is believed, then
it is known that $X$ is believed.  Property (kbm) says that belief is
monotonic: if an agent believes $X$, then she believes all
propositions $Y\supseteq X$ that follow from $X$.

\begin{definition}
  The language $\Lang_\KB$ of \emph{multi-agent knowledge and belief}
  is defined by the following grammar.
  \begin{eqnarray*}
    \phi & ::= & 
    \top \mid p \mid \neg\phi \mid \phi\land\phi \mid
    K_a\phi \mid B_a\phi
    \\
    &&
    \text{\footnotesize 
      $p\in\Prop$,
      $a\in A$
    }
  \end{eqnarray*}
  We adopt the usual abbreviations for other Boolean connectives and
  define the dual operators $\check K_a:=\lnot K_a\lnot$ and $\check
  B_a:=\lnot B_a\lnot$.
\end{definition}

We now turn to the definition of truth for the language $\Lang_\KB$.

\begin{definition}
  \label{definition:LKB-truth}
  Let $\M = (W,R,V,N)$ be an epistemic neighborhood model.  We define
  a binary truth relation $\modelsn$ between a pointed epistemic
  neighborhood model $(\M,w)$ and $\Lang_\KB$-formulas and a function
  $\semn{\cdot}^\M:\Lang_\KB\to \pow(W)$ as follows.
  \begin{eqnarray*} 
    \semn{\phi}^\M & := & \{v\in W\mid \M,v\modelsn\phi\}
    \\
    \M, w \modelsn p & \text{ iff } & p \in V(w) 
    \\
    \M, w \modelsn \neg \phi & \text{ iff } & \M, w \not\modelsn \phi 
    \\
    \M, w \modelsn \phi\land\psi  & \text{ iff } 
    & \M, w \modelsn \phi \text{ and } \M, w \modelsn \psi
    \\
    \M, w \modelsn K_a \phi  & \text{ iff } & 
    [w]_a\subseteq\semn{\phi}^\M
    \\
    \M, w \modelsn B_a \phi  & \text{ iff } &
    [w]_a\cap \semn{\phi}^\M \in N_a(w)
  \end{eqnarray*}
  Validity of $\phi\in\Lang_\KB$ in an epistemic neighborhood model
  $\M$, written $\M\modelsn\phi$, means that $\M,w\modelsn\phi$ for
  each world $w\in W$.  Validity of $\phi\in\Lang_\KB$, written
  $\modelsn\phi$, means that $\M\modelsn\phi$ for each epistemic
  neighborhood model $\M$.  For a class $\mathcal{C}$ of epistemic
  neighborhood models, we write $\mathcal{C}\modelsn\phi$ to mean that
  $\M\modelsn\phi$ for each $\M\in\mathcal{C}$.
\end{definition}

Intuitively, $K_a\phi$ is true at $w$ iff $\phi$ holds at all worlds
epistemically possible with respect to $w$, and $B_a\phi$ holds at $w$
iff the epistemically possible $\phi$-worlds make up a neighborhood of
$w$.  Note that it follows from this definition that the dual for
belief $\check{B}_a \phi$ is true at $w$ iff
$[w]_a\cap\semn{\neg\phi}^\M\notin N_a(w)$.

\section{Updates on Epistemic Neighborhood Models} 

Action models over this language are defined by: 

\begin{definition} 
  Given a language $L$, an \emph{epistemic neighborhood action model}
  is a structure $(\bar{W},\bar{R},\pre,\bar{N})$ satisfying the
  following.
  \begin{itemize}
  \item $\bar{W}$ is a nonempty finite set of whose members are called
    ``events.''

  \item $\bar{R}$ is a function that assigns to every agent $a \in A$
    an equivalence relation $\bar{R}_a$ on $\bar{E}$.

  \item $\pre:\bar{E}\to L$ is a \emph{precondition function} that
    maps each event to an $L$-formula.

  \item $\bar{N} : A\times \bar{E} \to \pow(\pow(\bar{E}))$ is a \emph{neighborhood
      function} that assigns to each agent $a\in A$ and event $e\in \bar{E}$
    a collection $\bar{N}_a(e)$ of sets of events, subject to the following
    conditions.
    \begin{description}
    \item[(kbc)] $\forall X \in \bar{N}_a(e) : X \subseteq [e]_a$.

    \item[(bf)] $\emptyset\notin \bar{N}_a(e)$.
      
    \item[(n)] $[e]_a\in \bar{N}_a(e)$.
      
    \item[(a)] $\forall d \in [e]_a : \bar{N}_a(d) = \bar{N}_a(e)$.

    \item[(kbm)] $\forall X \subseteq Y \subseteq [e]_a : \text{ if } X
      \in \bar{N}_a(e) \text{, then } Y \in \bar{N}_a(e)$.
    \end{description}
  \end{itemize} A \emph{pointed neighborhood action model} is a pair
  $(\A,e)$ consisting of an neighborhood action model $\A$ and an
  event $e$ in the set of events of $\A$.  We let $\Actm(L)$ denote
  the set of action models with preconditions in the language $L$, and
  we let $\Actm_*(L)$ denote the set of pointed action models with
  preconditions in the language $L$.
\end{definition}

\begin{definition}
  The language $\Lang_\KB^U$ of \emph{multi-agent knowledge and belief
    with updates} is an extension of $\Lang_\KB$ defined by the
  following recursive grammar.
  \begin{eqnarray*}
    \phi & ::= & 
    \top \mid p \mid \neg\phi \mid \phi\land\phi \mid
    K_a\phi \mid B_a\phi \mid [\A,e]\phi
    \\
    &&
    \text{\footnotesize 
      $p\in\Prop$,
      $a\in A$,
      $(\A,e)\in\Actm_*(\Lang_\KB^U)$
    }
  \end{eqnarray*}
\end{definition}

\begin{definition}
  \label{definition:LKBU-truth}
  Let $\M = (W,R,V,N)$ be an epistemic neighborhood model.  We define a
  binary truth relation $\modelsn$ between a pointed epistemic
  neighborhood model $(\M,w)$ and $\Lang_\KB^U$-formulas as the
  extension of the relation from Definition~\ref{definition:LKB-truth}
  obtained by adding the following clause.
  \begin{eqnarray*} 
    \M, w \modelsn [\A,e]\phi & \text{ iff } &
    \M, w \not\modelsn \pre(e) \text{ or }
    \M[\A], (w,e) \modelsn\phi
  \end{eqnarray*}
  where we define $\M[\A]:=(W',R',V',N')$ in terms of the action model
  $\A=(\bar{W},\bar{R},\pre,\bar{N})$ as follows:
  \begin{itemize}
  \item $W' := \{(w,e)\in W\times\bar{W}\mid \M , w \models \pre(e) \}$,

  \item $R'_a:=
    \{ ((w,e),(v,f))\in W'\times W' \mid
    (w,v)\in R_a \text{ and } (e,f)\in\bar{R}_a \}$,

  \item $V'(w,e) := V(w)$, and

  \item $N': A\times W' \to \pow(\pow(W'))$ is given by 
    \[
    N'_a(w,e) = \{ (X \times Y) \cap W' \mid X \in N_a (w) \text{ and } Y \in
    \bar{N}_a(e) \} \enspace.
    \]
  \end{itemize}
  Note that, as in the definition of $N'$ above, we often write a
  function $f$ applied to a pair $(a,b)$ as $f(a,b)$ instead of the
  more cumbersome $f((a,b))$.  We say that a pointed neighborhood
  action model $(\A,e)$ is \emph{executable} at $(\M,w)$ to mean that
  $\M,w\modelsn\pre(e)$.  We verify in
  Theorem~\ref{theorem:well-defined} that $\M[\A]$ is an epistemic
  neighborhood model whenever $(\A,e)$ is executable at $(\M,w)$.
  This ensures that the above is well-defined.  The various notions of
  validity for $\Lang_\KB^U$-formulas are defined as were the various
  notions for $\Lang_\KB$-formulas.
\end{definition}

\begin{theorem}
  \label{theorem:well-defined}
  Let $(\M,w)$ and $(\A,e)$ be given as in
  Definition~\ref{definition:LKBU-truth}.  If $(\A,e)$ is executable
  at $(\M,w)$, then $\M[\A]$ is an epistemic neighborhood model.
\end{theorem}
\begin{proof}
  Since $(\A,e)$ is executable at $(\M,w)$, it follows that $W'$ is
  nonempty.  Since $R_a$ and $\bar{R}_a$ are equivalence relations,
  $R'_a$ is an equivalence relation.  So all that remains is to verify
  that $N'$ satisfies (kbc), (bf), (n), (a), and (kbm).

  For (kbc), let $Z \in N'_a(w,e)$. Then $Z$ has the form $(X \times Y)
  \cap W'$ for some $X \in N_a(w), Y \in \bar{N}_a(e)$. Since $\M$ and
  $\A$ both satisfy (kbc), we have $X \subseteq [w]_a$ and $Y \subseteq
  [e]_a$.  By the product construction on $R'_a$, we have $[(w,e)]_a =
  ([w]_a \times [e]_a) \cap W'$. It follows that $Z \subseteq
  [(w,e)]_a$.

  For (bf), the result follows because $W'$ is nonempty.

  For (n), take $(w,e) \in W'$. Then $[(w,e)]_a$ equals $([w]_a \times
  [e]_a) \cap W'$.  Since $\M$ and $\A$ satisfy (n), we have $([w]_a
  \times [e]_a) \cap W' \in N'_a(w,e)$.

  For (a), take $(w',e') \in [(w,e)]_a$. Then $w' \in [w]_a$ and $e'
  \in [e]_a$.  Since $\M$ and $\A$ satisfy (a), $N_a(w') = N_a(w)$ and
  $\bar{N}_a(e') = \bar{N}_a(e)$.  Therefore $N'_a (w',e') = N'_a
  (w,e)$.
 
  For (kbm), take $X \subseteq Y \subseteq [(w,e)]_a$ and suppose $X \in
  N_a(w,e)$. Then there are $U \in N_a(w)$ and $V \in N_a(e)$
  satisfying $N_a(w,e) = (U\times V) \cap W'$, and therefore $Y
  \supseteq (U\times V) \cap W'$. Since $\M$ and $\A$ satisfy (kbm), $Y
  \in N_a(w,e)$.
\end{proof}

What the theorem tells us is that the situation here is markedly
different from the situation of updating KD45 doxastic models with
KD45 event models. Since the seriality requirement on KD45 models is
not universal Horn, it is not preserved under product update.
Therefore, updating a static KD45 model with an update KD45 model does
not guarantee a result where the accessibilities are again KD45.

Neighborhood models for belief are better behaved under product update 
than KD45 models for belief. 

\section{Axiomatic Theories}

\subsection{The Basic Theory \texorpdfstring{$\KB$}{KB}}

\begin{definition}
  The theory $\KB$ is defined in Table~\ref{table:KB}.  All
  formulas are in the language $\Lang_\KB$.
\end{definition}

\begin{table}[ht]
  \begin{center}
    \textsc{Axiom Schemes}\\[.4em]
    \renewcommand{\arraystretch}{1.3}
    \begin{tabular}[t]{cl}
      (CL) &
      Schemes of Classical Propositional Logic
      \\
      (KS5) &
      $\mathsf{S5}$ axiom schemes for each $K_a$
      \\
      (KBC) &
      $K_a\phi\to B_a\phi$
      \\
      (BF) &
      $\lnot B_a\bot$
      \\
      (N) &
      $B_a\top$
      \\
      (Ap) &
      $B_a\phi\to K_aB_a\phi$
      \\
      (An) &
      $\lnot B_a\phi\to K_a\lnot B_a\phi$
      \\
      (KBM) &
      $K_a(\phi\to\psi)\to(B_a\phi\to B_a\psi)$
    \end{tabular}
    \renewcommand{\arraystretch}{1.0}
    \\[1em]
    \textsc{Rules}\vspace{-.5em}
    \[
    \begin{array}{c}
      \phi\to\psi \quad \phi
      \\\hline
      \psi
    \end{array}\;\text{\footnotesize(MP)}
    \qquad
    \begin{array}{c}
      \phi
      \\\hline
      K_a\phi
    \end{array}\;\text{\footnotesize(MN)}
    \]
  \end{center}
  \caption{The theory $\KB$}
  \label{table:KB}
\end{table}

The following result, originally proved in \cite{EijRen14:BEPL}, shows
that if we restrict attention to provable statements whose only
modality is single-agent belief $B_a\phi$, then $\KB$ is an extension
of the minimal modal logic $\mathsf{EMN45}+\lnot
B_a\bot=\mathsf{EMN45}+(\text{BF})$ obtained by adding
$\mathsf{S5}$-knowledge and the knowledge-belief connection principles
(KBC), (Ap), (An), and (KBM).\footnote{$\mathsf{EMN45}+(\text{BF})$ is
  the logic of single-agent belief (without knowledge) having Schemes
  (CL) (Table~\ref{table:KB}), M
  (Prop.~\ref{prop:KBgt-derivables}\eqref{derivables:Band-andB}), (N)
  (Table~\ref{table:KB}), 4
  (Prop.~\ref{prop:KBgt-derivables}\eqref{derivables:pos-belief}), 5
  (Prop.~\ref{prop:KBgt-derivables}\eqref{derivables:neg-belief}), and
  (BF) (Table~\ref{table:KB}) along with Rules (MP)
  (Table~\ref{table:KB}) and RE
  (Prop.~\ref{prop:KBgt-derivables}\eqref{derivables:RE}). This is a
  ``monotonic'' system of modal logic satisfying positive and negative
  belief introspection (4 and 5) and the property (BF) that falsehood
  $\bot$ is not believed. See \cite[Ch.~8]{Chellas:ml} for details on
  naming minimal modal logics.}

\begin{proposition}[$\KB$ Derivables; \cite{EijRen14:BEPL}]
  \label{prop:KBgt-derivables}
  We have each of the following.
  \begin{enumerate}
  \item $\KB\vdash B_a(\phi\land\psi)\to(B_a\phi\land B_a\psi)$.
    \label{derivables:Band-andB}

    This is ``Scheme M'' \cite[Ch.~8]{Chellas:ml}.

  \item $\KB\vdash K_a\phi\land B_a\psi\to B_a(\phi\land\psi)$.
    \label{derivables:andB-Band}

    If the antecedent $K_a\phi$ were replaced by $B_a\phi$, then we
    would obtain ``Scheme C'' \cite[Ch.~8]{Chellas:ml}.  So we do not
    have Scheme C outright but instead a knowledge-weakened version:
    in order to conclude belief of a conjunction from belief of one of
    the conjuncts, the other conjunct must be known (and not merely
    believed, as is required by the stronger, non-$\KB$-provable
    Scheme C).

  \item $\KB\vdash K_a(\phi\to\psi)\to(\check B_a\phi\to\check B_a\psi)$.
    \label{derivables:check-M}

    This is the dual version of our (KBM).

    \item $\KB\vdash B_a\phi\to B_aB_a\phi$.
    \label{derivables:pos-belief}

    This is ``Scheme 4'' \cite[Ch.~8]{Chellas:ml}.

  \item $\KB\vdash \lnot B_a\phi\to B_a\lnot B_a\phi$.
    \label{derivables:neg-belief}

    This is ``Scheme 5'' \cite[Ch.~8]{Chellas:ml}.

  \item $\KB\vdash B_a\phi\leftrightarrow K_aB_a\phi$.
    \label{derivables:B-KB}

    This says that belief and knowledge of belief are equivalent.

  \item $\KB\vdash \lnot B_a\phi\leftrightarrow K_a\lnot B_a\phi$.
    \label{derivables:nB-KnB}

    This says that non-belief and knowledge of non-belief are equivalent.

  \item $\KB\vdash\phi$ implies $\KB\vdash B_a\phi$.
    \label{derivables:B-nec}

    This is the rule of Modus Ponens (or Modal Necessitation),
    sometimes called ``Rule RN'' \cite[Ch.~8]{Chellas:ml}.

  \item $\KB\vdash\phi\to\psi$ implies $\KB\vdash B_a\phi\to B_a\psi$.
    \label{derivables:Bimp}

    This is ``Rule RM'' \cite[Ch.~8]{Chellas:ml}.

  \item $\KB\vdash\phi\to\psi$ implies $\KB\vdash\check
    B_a\phi\to\check B_a\psi$.
    \label{derivables:check-Bimp}
    
    This is the dual version of RM.
    
  \item $\KB\vdash\phi\leftrightarrow\psi$ implies
    $\KB\vdash B_a\phi\leftrightarrow B_a\psi$.
    \label{derivables:RE}

    This is ``Rule RE'' \cite[Ch.~8]{Chellas:ml}.

  \item $\KB\vdash\phi\to\bot$ implies $\KB\vdash\lnot
    B_a\phi$. \label{derivables:GBF}

    This says that no self-contradictory sentence is believed.  This
    may be viewed as a certain generalization of (BF)
    (Table~\ref{table:KB}).
  \end{enumerate}
\end{proposition}

\begin{theorem}[$\KB$ Soundness and Completeness; \cite{EijRen14:BEPL}]
  \label{theorem:KB-neighborhood-soundness}
  $\KB$ is sound and complete with respect to the class $\mathcal{C}$ of
  epistemic neighborhood models:
  \[
  \forall\phi\in\Lang_\KB:\quad\KB\vdash\phi
  \quad\Leftrightarrow\quad
  \mathcal{C}\modelsn\phi
  \enspace.
  \]
  Also, every satisfiable $\KB$-formula is satisfiable in a finite
  epistemic neighborhood model.
\end{theorem}

\subsection{The Theory with Updates
  \texorpdfstring{$\KB^U$}{KBU}}

\begin{definition}
  The theory $\KB^U$ is defined in Table~\ref{table:KBU}.  All
  formulas are in the language $\Lang_\KB^U$. Schemes (RA), (RN), (RC),
  (RK), and (RB) are called \emph{reduction schemes} (or
  \emph{reduction axioms}).
\end{definition}

\begin{table}[ht]
  \begin{center}
    \textsc{Axiom Schemes}\\[.4em]
    \renewcommand{\arraystretch}{1.3}
    \begin{tabular}[t]{cl}
      ($\KB$) & 
      Schemes of $\KB$ (Table~\ref{table:KB})
      \\
      (RA) &
      $[\A,e]p \leftrightarrow(\,\pre(e)\to p\,)$ for
      $p\in\{\top\}\cup\Prop$
      \\
      (RN) &
      $[\A,e]\lnot\phi \leftrightarrow(\,\pre(e)\to \lnot[\A,e]\phi\,)$
      \\
      (RC) &
      $[\A,e](\phi\land\psi) \leftrightarrow(\,[\A,e]\phi\land[\A,e]\psi\,)$
      \\
      (RK) &
      $\textstyle
      [\A,e]K_a\phi \leftrightarrow
      (\,\pre(e)\to \bigwedge_{(e,f)\in\bar{R}_a}K_a[\A,f]\phi\,)$
      \\
      (RB) &
      $\textstyle
      [\A,e]B_a\phi \leftrightarrow
      (\,\pre(e)\to \bigvee_{Y\in\bar{N}_a(e)}
      B_a(\bigwedge_{f\in
        Y}[\A,f]\phi)\,)$
    \end{tabular}
    \renewcommand{\arraystretch}{1.0}
    \\[1em]
    \textsc{Rules}\vspace{-.5em}
    \[
    \begin{array}{c}
      \phi\to\psi \quad \phi
      \\\hline
      \psi
    \end{array}\;\text{\footnotesize(MP)}
    \qquad
    \begin{array}{c}
      \phi
      \\\hline
      K_a\phi
    \end{array}\;\text{\footnotesize(MN)}
    \qquad
    \begin{array}{c}
      \phi
      \\\hline
      [\A,e]\phi
    \end{array}\;\text{\footnotesize(UN)}
    \]
  \end{center}
  \caption{The theory $\KB^U$}
  \label{table:KBU}
\end{table}

\begin{theorem}[Reduction]
  \label{theorem:reduction}
  For each $\phi\in\Lang_\KB^U$, there exists
  $\phi^\circ\in\Lang_\KB$ such that
  \[
  \KB^U\vdash\phi\leftrightarrow\phi^\circ
  \enspace.
  \]
\end{theorem}
\begin{proof}
  Let $L_0:=\Lang_\KB$.  Once $L_i$ is defined, define $L_{i+1}$ by
  \begin{eqnarray*}
    \phi_{i+1} & ::= & 
    \phi_i \mid \neg\phi_{i+1} \mid \phi_{i+1}\land\phi_{i+1} \mid
    K_a\phi_{i+1} \mid B_a\phi_{i+1} \mid [\A,e]\phi_{i+1}
    \\
    &&
    \text{\footnotesize 
      $\phi_i\in L_i$,
      $a\in A$,
      $(\A,e)\in\Actm_*(L_i)$
    }
  \end{eqnarray*}
  Observe that $i\leq j$ implies $L_i\subseteq
  L_j\subseteq\Lang_\KB^U$ and
  $\Actm_*(L_i)\subseteq\Actm_*(L_j)\subseteq\Actm_*(\Lang_\KB^U)$.  It
  follows that $\Lang_\KB^U=\bigcup_{i\in\mathbb{N}}L_i$.  We define a
  dictionary ordering on $\Lang_\KB^U$ in a few steps.  First, for
  each $\phi\in\Lang_\KB^U$, let $d(\phi)$ denote the
  \emph{non--precondition-recursing formation depth} of $\phi$ and let
  $D(\phi)$ denote the \emph{post-action non--precondition-recursing
    formation depth} of $\phi$.  These are defined as follows.
  \[
  \renewcommand{\arraystretch}{1.3}
  \begin{array}[t]{rcl}
    d(p) &:=& 0 \text{ for } p\in\{\top\}\cup\Prop \\
    d(\lnot\phi) &:=& 1 + d(\phi) \\
    d(\phi\land\psi) &:=& 1 + \max\{d(\phi),d(\psi)\} \\
    d(K_a\phi) &:=& 1 + d(\phi) \\
    d(B_a\phi) &:=& 1 + d(\phi) \\
    d([\A,e]\phi) &:=& 1 + d(\phi)
  \end{array}
  \begin{array}[t]{rcl}
    D(p) &:=& 0 \text{ for } p\in\{\top\}\cup\Prop \\
    D(\lnot\phi) &:=& D(\phi) \\
    D(\phi\land\psi) &:=& \max\{D(\phi),D(\psi)\} \\
    D(K_a\phi) &:=& D(\phi) \\
    D(B_a\phi) &:=& D(\phi) \\
    D([\A,e]\phi) &:=& d(\phi)
  \end{array}
  \renewcommand{\arraystretch}{1.0}
  \]
  $d(\phi)$ is the maximum number of recursive steps required to
  construct $\phi$ without counting steps required to construct
  preconditions.  $D(\phi)$ is the maximum number of recursive steps
  required to construct that part of $\phi$ that occurs just after a
  neighborhood action model modality is first encountered in the
  recursive breakdown of $\phi$.  We let $\ell(\phi)$ denote the
  \emph{recursion level} of $\phi$: this is the smallest
  $i\in\mathbb{N}$ satisfying $\phi\in L_i$.  Finally, we order
  $\Lang_\KB^U$-formulas as follows: $\phi<\psi$ means that either
  $\ell(\phi)<\ell(\psi)$ or else both $\ell(\phi)=\ell(\psi)$ and
  $D(\phi)<D(\psi)$.  The ordering $<$ well-orders $\Lang_\KB^U$.

  It is not difficult to prove that for each
  $(\A,e)\in\Actm_*(\Lang_\KB^U)$ and $\phi\in\Lang_\KB^U$, we have
  $\ell(\pre(e))<\ell([\A,e]\phi)$ and therefore $\pre(e)<[\A,e]\phi$.
  So for each reduction axiom, if $\phi_{\mathsf{L}}$ is the formula
  on the left side of the biconditional and $\phi_{\mathsf{R}}$ is the
  formula on the right side of the biconditional, then it can be shown
  that $\phi_{\mathsf{R}}<\phi_{\mathsf{L}}$.  We therefore may define
  $\circ:\Lang_\KB^U\to\Lang_\KB$ as follows:
  \[
  \renewcommand{\arraystretch}{1.3}
  \begin{array}[t]{lcl}
    p^\circ &:=& p \text{ for } p\in\{\top\}\cup\Prop \\
    (\lnot\phi)^\circ &:=& \lnot\phi^\circ \\
    (\phi\land\psi)^\circ &:=& \phi^\circ\land\psi^\circ \\
    (K_a\phi)^\circ &:=& K_a\phi^\circ \\
    (B_a\phi)^\circ &:=& B_a\phi^\circ \\
    ([\A,e]p)^\circ &:=& 
    \textstyle
    \pre(e)^\circ\to p \text{ for }  p\in\{\top\}\cup\Prop 
    \\
    ([\A,e]\lnot\phi)^\circ &:=& 
    \pre(e)^\circ\to \lnot([\A,e]\phi)^\circ
    \\
    ([\A,e](\phi\land\psi))^\circ &:=&
    ([\A,e]\phi)^\circ\land([\A,e]\psi)^\circ
    \\
    ([\A,e]K_a\phi)^\circ &:=& 
    \textstyle
    \pre(e)^\circ\to \bigwedge_{(e,f)\in\bar{R}_a}K_a([\A,e]\phi)^\circ
    \\
    ([\A,e]B_a\phi)^\circ &:=& 
    \textstyle
    \pre(e)^\circ\to
    \bigvee_{Y\in\bar{N}_a(e)} B_a(\bigwedge_{f\in Y}([\A,f]\phi)^\circ)
    \\
    ([\A,e][\B,f]\phi)^\circ &:=&
    ([\A,e]([\B,f]\phi)^\circ)^\circ
  \end{array}
  \renewcommand{\arraystretch}{1.0}
  \]
  Making use of the well-ordering $<$, the reduction axioms,
  classical reasoning, and the rule (UN), the result follows.
\end{proof}

\begin{theorem}[$\KB^U$ Soundness and Completeness]
  $\KB^U$ is sound and complete with respect to the class
  $\mathcal{C}$ of epistemic neighborhood models:
  \[
  \forall\phi\in\Lang_\KB^U:\quad\KB^U\vdash\phi
  \quad\Leftrightarrow\quad
  \mathcal{C}\modelsn\phi
  \enspace.
  \]
\end{theorem}
\begin{proof}
  \XXXcomment{BR: insert proof of (RB) soundness. Fix axiom (RB).}
  For soundness, it suffices by
  Theorem~\ref{theorem:KB-neighborhood-soundness} to show that each of
  the reduction axioms is sound and the rule (UN) preserves soundness.
  Most arguments for most of the reduction axioms and for (UN) are
  standard \cite{DitHoeKoo06:del}.  We only consider the novel case:
  the reduction axiom (RB).  That is, we argue that
  \[
  \textstyle \modelsn 
  \begin{array}[t]{l}
    [\A,e]B_a\phi \leftrightarrow 
    (\,\pre(e)\to
    \bigvee_{Y\in\bar{N}_a(e)} B_a(\bigwedge_{f\in Y}[\A,f]\phi)\,) \enspace.
  \end{array}
  \]
  Right to left: assume $M,w\modelsn[\A,e]B_a\phi$.  If
  $\M,w\not\modelsn\pre(e)$, then the right hand side of the above
  biconditional follows, so let us assume further that
  $\M,w\modelsn\pre(e)$. By Definition~\ref{definition:LKBU-truth}, our
  assumption $\M,w\modelsn[\A,e]B_a\phi$ implies that there exists
  $X\in N_a(w)$ and $Y\in\bar{N}_a(e)$ such that
  \[
  \semn{\phi}\cap[w,e]=(X\times Y)\cap W'\enspace,
  \]
  where $W'$ is given as in Definition~\ref{definition:LKBU-truth}.

  % XXX; fill in argument

  Completeness follows by Reduction (Theorem~\ref{theorem:reduction})
  and soundness and completeness of $\KB$
  (Theorem~\ref{theorem:KB-neighborhood-soundness}).
\end{proof}

\section{Examples} 

\ldots 

\section{Some Comparisons} 

\ldots 

\section{Conclusion and Further Work} 

\ldots 

\paragraph{Acknowledgements} 

Thanks to \ldots 

\bibliographystyle{alpha}
\bibliography{NU}

\end{document} 
